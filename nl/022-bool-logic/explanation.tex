\paragraph{Logische operaties}

Het is mogelijk om de waarden \texttt{true} en \texttt{false} te combineren tot meer complexe logische expressies. Zo zijn er de logische operaties \texttt{\&\&} (and) en \texttt{||} (or). Dit zijn de zogeheten waarheidstabellen voor deze operaties:

\begin{center}
  \ttfamily
  \begin{tabular}{r@{ \&\& }l@{\qquad}l}
    \multicolumn{2}{l}{\normalfont expressie} & {\normalfont geeft} \\
    \midrule
    true & true   & true \\
    true & false & false \\
    false & true  & false \\
    false & false  & false \\
    \midrule
  \end{tabular}
  \qquad
  \begin{tabular}{r@{ || }l@{\qquad}l}
    \multicolumn{2}{l}{\normalfont expressie} & {\normalfont geeft} \\
    \midrule
    true  & true   & true \\
    true  & false  & true \\
    false & true   & true \\
    false & false  & false \\
    \midrule
  \end{tabular}
\end{center}

\paragraph{Ontkenning}

De uitkomst van een bewering kan ook ``omgedraaid'' worden. Zet er dan \texttt{!} (not) voor:

\begin{center}
  \texttt{!true} \quad  geeft \quad  \texttt{false} \qquad  en \qquad  \texttt{!false} \quad geeft \quad \texttt{true}
\end{center}

\paragraph{Prioriteitsregels}

Ook \texttt{\&\&} en \texttt{||} hebben hun plek in de prioriteitsregels:

\begin{enumerate}
  \item eerst de expressies tussen haakjes
  \item dan rekenkundige operaties zoals \texttt{+} en \texttt{*}
  \item dan beweringen zoals \texttt{>}, \texttt{<}, \texttt{==}, \texttt{!=}, \texttt{>=}, \texttt{<=}
  \item \texttt{!}
  \item \texttt{\&\&}
  \item \texttt{||}
\end{enumerate}
