In de vorige paragrafen heb je steeds expressies \emph{geïnterpreteerd}, de uitkomst berekend op basis van de rekenregels. Nu vragen we je om zelf expressies te schrijven. Dit is wat creatiever, en het kan soms moeilijk zijn om op het juiste idee te komen. Blader dan eens terug naar de vorige delen voor inspiratie.

Als je zo'n expressie schrijft, kun je deze zelf nalopen met behulp van de regels die je kent. Schrijf je expressie op, en probeer deze te interpreteren. Je kunt ook een aantal ``testgevallen'' bedenken die je voor het getal \texttt{n} invult, om te kijken of de uitkomst naar verwachting is.

\begin{exercise}
    \begin{longtasks}[resume=true](1)
        \task
        Schrijf een expressie die test of een getal \texttt{n} kleiner is dan 5.
        \task
        Schrijf een expressie die test of een getal \texttt{n} tussen 5 en 10 zit, waarbij 5 en 10 zelf \'{o}\'{o}k mogen.
        \task
        Schrijf een expressie die test of een getal \texttt{n} deelbaar is door 3.
        \task
        Schrijf een expressie die test of een getal \texttt{n} \emph{even} is.
        \task
        Schrijf een expressie die test of een getal \texttt{n} \emph{oneven} is.
        \task
        Voor welke waarden van \texttt{n} is de bewering \texttt{!(n > 1 \&\& !(n > 3)) \&\& true} waar?
        \task
        De bewering uit de vorige vraag kan eenvoudiger worden geschreven. Doe dit, en controleer je vereenvoudigde versie door te kijken of deze nog steeds waar is voor dezelfde waarden van \texttt{n}.
    \end{longtasks}
\end{exercise}

\begin{solution}
    Geen antwoorden voor paragraaf 2.3.
\end{solution}


% \begin{solution}
% \texttt{n} < 5
% \end{solution}

% \begin{solution}
% 5 <= \texttt{n} <= 10
% \end{solution}

% \begin{solution}
% \texttt{n} % 3 == 0
% \end{solution}

% \begin{solution}
% \texttt{n} % 2 == 0
% \end{solution}

% \begin{solution}
% \texttt{n} % 2 != 0
% \end{solution}

% \begin{solution}
% n <= 1 \& n > 3
% \end{solution}

% \begin{solution}
% !(n >= 1 \&\& n < 3)
% \end{solution}
