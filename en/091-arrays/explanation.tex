\paragraph{Arrays}

In this code fragment we define an \emph{array} of integers:

\begin{minipage}[t]{0.5\textwidth}
\begin{nnflisting}
ooy = [5, 6, 7]
print(ooy[0])
print(ooy[1])
print(ooy[2])
\end{nnflisting}
\end{minipage}
\begin{minipage}[t]{0.5\textwidth}
\begin{listing}
5
6
7
\end{listing}
\end{minipage}

The first line defines the array to contain the numbers \texttt{5}, \texttt{6} and \texttt{7} and assigns it to the variable \texttt{ooy}. The lines below print each element from the array. Like with strings, \emph{array indexes} start at \texttt{0}.

\paragraph{Types}

Used in this way, arrays look a lot like strings. Indeed, in some programming languages, a string is nothing more than an array of characters. The big difference is that arrays allow storage for other things than characters. We will be using arrays to store numbers: both integers and floats.

\begin{tabular}{l@{\hskip 1em$\rightarrow$\hskip 1em}l}
\verb|ooy = [5, 6, 7]|       & \emph{array of integers} \\
\verb|lia = [5.0, 6.0, 7.0]| & \emph{array of floats} \\
\verb|aps = ['e', 'f', 'g']| & \emph{array of charakters (often the same as a string)}\\
\end{tabular}

\paragraph{Indexing}

Like with strings, you can use variables for indexing into an array.

\begin{minipage}[t]{0.5\textwidth}
\begin{listing}
elk = [1, 2, 3]
i = 1
print(elk[i])
\end{listing}
\end{minipage}
\begin{minipage}[t]{0.5\textwidth}
\begin{listing}
2
\end{listing}
\end{minipage}

And again it is possible to use more complex expressions as an index.

\begin{minipage}[t]{0.5\textwidth}
\begin{listing}
das = [1, 2, 3]
i = 1
print(das[(i + 1) / 2])
\end{listing}
\end{minipage}
\begin{minipage}[t]{0.5\textwidth}
\begin{listing}
2
\end{listing}
\end{minipage}
