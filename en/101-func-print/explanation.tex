\paragraph{Defining a function}

There is a separation between the fuction \emph{definition}, which describes what actions the function will perform, and the function \emph{call}, which signals that our function should be run. This means that as we define a function, it will not automatically be run, allowing us to postpone running it until we need it. This, in turn, allows us to call a function multiple times in different parts of our programs. In this book, a function definition will look like this:

\begin{verbatim}
<type> <name of function>():
    <function body>
\end{verbatim}

For now, we will use \texttt{void} for the \texttt{<type>}. The word \texttt{void} signals that the function is intended to perform some action, we will see functions that calculate something later. One example of a function that performs an action is a printing function:

\begin{verbatim}
void print_reassuring_message():
    print("I'm still here!")
\end{verbatim}

\paragraph{Calling a function}

Calling a function can be done from anywhere in the program as long as it has been \emph{defined} previously. Calling a function looks like this:

\begin{verbatim}
<name of function>()
\end{verbatim}

As you can see, the parentheses \texttt{()} are an important part of the function definition, as well as of the function call. Most programming languages use these to discern functions from other elements of the program.

% TODO potential end note: the Ruby language makes () optional even for function calls, which means that the names of functions must be even more discerning than in other languages

\paragraph{Tracing} Function calls can be traced, but because the definition and calls to functions are separate, we need a clear notation. Say we take the program below. The execution of the program starts at the first line that is not a function definition (in our case, the last line). On this line, the function \texttt{baz} is called. In the function \texttt{baz}, \texttt{bar} is called, which we can then ``jump''  to. The key is to strictly follow the top-down sequence of statements, unless there is a function call.

\includegraphics[width=.4\textwidth]{1-trace-calls.jpeg}

In our trace we draw lines next to the functions to have a clear separation, and we number the lines in order of execution. We see that ``fly'' is printed before ``rainbow''.
