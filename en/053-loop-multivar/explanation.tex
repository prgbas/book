\paragraph{Tracing multiple variables with loops}

Consider the following code fragment:

\begin{nnflisting}
a = 0
b = 0
while(a < 2)
    a = a + 1
    b = b + 1
a = 42
\end{nnflisting}

The accompanying trace looks like this;

\begin{tracelist}[l|ccccccccccc]
  & \texttt{1} & \texttt{2} & \texttt{3} &  \texttt{4} &
                          \texttt{5} & \texttt{3} & \texttt{4} &  \texttt{5} &
                                                    \texttt{3} & \texttt{6}  \\ \hline
\\[-1em]
var \texttt{a} & \fbox{0} & 0 & 0 & \fbox{1} & 1 & 1 & \fbox{2} & 2 & 2 & \fbox{42} \\
var \texttt{b} & & \fbox{0} & 0 & 0 & \fbox{1} & 1 & 1 & \fbox{2} & 2 & 2 \\
a < 2 & & & true & & & true & & & \fbox{false}
\end{tracelist}

\paragraph{Incrementing}

Loops often use a \textbf{counter}, a variable that is incremented by 1 in each loop step. Many programming languages have a special operator to do this: \texttt{++}. The statement \texttt{i++} is equivalent to \texttt{i = i + 1}. Decrementing also has a shortcut: \texttt{i--}. This command performs the same operator as \texttt{i = i - 1}.
