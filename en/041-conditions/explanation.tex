\paragraph{Conditional statements}

The following code fragment contains a conditional statement, which starts with the keyword \texttt{if}. There is a requirement, or \emph{condition}, which is \texttt{a < 10}. The statement on the following line has been indented (moved to the right) which will show that the statement is dependent on the \texttt{if}.

\begin{nnflisting}
a = 0
if(a < 10)
    a = a + 10
a = a - 1
\end{nnflisting}

We start the program on line 1 with the value \texttt{a = 0}. This means that on line 2, the condition \texttt{a < 10} has indeed been met. The result is that the instruction \texttt{a = a + 10} will be executed too. Line 4 is printed all the way to the left, which means that it is not a part of the conditional statement: it will be executed regardless. This means that the program finishes with the variable \texttt{a} having a value \texttt{9}.

\paragraph{Tracing}

Like in the previous chapter we can trace the state of the variables. A trace for the previous code fragment looks like this:

\begin{tracelist-left}[lcccc]
  & 1 & \texttt{2} & \texttt{3} & \texttt{4} \\
\hline
\\[-1em]
a & \fbox{0} & 0 & \fbox{10} & \fbox{9}
\end{tracelist-left}

In the top row we print the program's line numbers. In the row below, we keep track of the variable \texttt{a}. In exactly three places, the value of \texttt{a} has a box: these are the lines where the value of the variable changed while tracing.

\paragraph{Failing condition}

In the code fragment below, we slightly changed the condition. In this case, at the moment of evaluating the condition, it yields \texttt{false}. Because the condition has not been met, the dependent line will be skipped. This means that the program finishes with the variable \texttt{a} having a value \texttt{-1}.

\begin{nnflisting}
a = 0
if(a > 10)
    a = a + 10
a = a - 1
\end{nnflisting}

And we skip line 3 in the trace:

\begin{tracelist-left}[lccccccc]
  & \texttt{1} & \texttt{2} &  \texttt{4} \\ \hline
\\[-1em]
\texttt{a} & \fbox{\texttt{0}} & \texttt{0} & \fbox{\texttt{-1}}
\end{tracelist-left}
