\paragraph{More complex conditions}

In this paragraph we reintroduce more complex conditions that are combinations of propositions with logical connectors. Below you'll find a summary of the rules that apply.

\paragraph{Propositions}
\begin{center}
\begin{tabular}{ll}
\textbf{operation} &        \textbf{meaning}\\
\midrule
      \texttt{==} &               equals\\
      \texttt{!=} &          does not equal\\
       \texttt{<} &              is smaller than\\
       \texttt{>} &               is larger than\\
      \texttt{<=} &  is smaller than or equal to\\
      \texttt{>=} &  is larger than or equal to\\
\midrule
\end{tabular}
\end{center}

\paragraph{Logische combinaties}
\begin{center}
  \ttfamily
  \begin{tabular}{r@{ \&\& }l@{\qquad}l}
    \multicolumn{2}{l}{\normalfont expression} & {\normalfont yields} \\
    \midrule
    true & true   & true \\
    true & false & false \\
    false & true  & false \\
    false & false  & false \\
    \midrule
  \end{tabular}
  \qquad
  \begin{tabular}{r@{ || }l@{\qquad}l}
    \multicolumn{2}{l}{\normalfont expression} & {\normalfont yields} \\
    \midrule
    true  & true   & true \\
    true  & false  & true \\
    false & true   & true \\
    false & false  & false \\
    \midrule
  \end{tabular}
\end{center}

\paragraph{Ontkenning}

\begin{center}
  \ttfamily
  \begin{tabular}{l@{\qquad}l}
    {\normalfont expression} & {\normalfont yields} \\
    \midrule
    !true   & false \\
    !false  & true \\
    \midrule
  \end{tabular}
\end{center}

The rules of precedence also apply like earlier.
