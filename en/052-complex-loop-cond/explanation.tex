\paragraph{Complexe condities}
In de opgaven hiernaast gebruiken weer complexere voorwaarden, bestaande uit beweringen en logische combinaties. Hieronder sommen we de regels voor zulke samengestelde beweringen nog eens op.

\paragraph{Beweringen}
\begin{center}
\begin{tabular}{ll}
\textbf{operatie} &        \textbf{betekenis}\\
\midrule
      \texttt{==} &               is gelijk aan\\
      \texttt{!=} &          is niet gelijk aan\\
       \texttt{<} &              is kleiner dan\\
       \texttt{>} &               is groter dan\\
      \texttt{<=} & is kleiner dan of gelijk aan\\
      \texttt{>=} & is groter dan of gelijk aan\\
\midrule
\end{tabular}
\end{center}

\paragraph{Logische combinaties}
\begin{center}
  \ttfamily
  \begin{tabular}{r@{ \&\& }l@{\qquad}l}
    \multicolumn{2}{l}{\normalfont expressie} & {\normalfont geeft} \\
    \midrule
    true & false & false \\
    true & true   & true \\
    false & true  & false \\
    false & false  & false \\
    \midrule
  \end{tabular}
  \qquad
  \begin{tabular}{r@{ || }l@{\qquad}l}
    \multicolumn{2}{l}{\normalfont expressie} & {\normalfont geeft} \\
    \midrule
    true  & false  & true \\
    true  & true   & true \\
    false & true   & true \\
    false & false  & false \\
    \midrule
  \end{tabular}
\end{center}

\paragraph{Ontkenning}
\begin{center}
  \texttt{!true} \quad  geeft \quad  \texttt{false} \qquad  en \qquad  \texttt{!false} \quad geeft \quad \texttt{true}
\end{center}

Vergeet bovendien niet de prioriteitsregels voor het rekenen.
