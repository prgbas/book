\paragraph{Propositions are true or false}

For propositions like \texttt{4\,==\,5} we can decide if they are \emph{true}. The number \texttt{4} does not equal \texttt{5}, which makes this proposition not true, or \emph{false}. There are six important ways to formulate a proposition:

\begin{center}
\begin{tabular}{ll}
\textbf{operator} &        \textbf{meaning}\\
\midrule
      \texttt{==} &               equals\\
      \texttt{!=} &          does not equal\\
       \texttt{<} &              is smaller than\\
       \texttt{>} &               is larger than\\
      \texttt{<=} &  is smaller than or equal to\\
      \texttt{>=} &  is larger than or equal to\\
\midrule
\end{tabular}
\end{center}

Each proposition formulated using one of these operators can evaluate to one of two values: \texttt{true} or \texttt{false}. These values are called \emph{booleans}\footnote{After George Boole, who appears to be the first to develop an algebraic system for logic in the 19th century.}. Expressions that evaluate to \texttt{true} or \texttt{false} are called \emph{boolean expressions}.

\paragraph{Precedence rules}

Like with arithmetic operations, propositional operations are subject to rules of precedence.

\begin{enumerate}
    \item any parts of the expression contained between parentheses come first
    \item then come the arithmetic operators \texttt{*} and \texttt{+} (note the rules that apply between those!)
    \item and finally the operations \texttt{>}, \texttt{<}, \texttt{==}, \texttt{!=}, \texttt{>=}, \texttt{<=} will be performed
\end{enumerate}

\paragraph{Mixing floats and integers}

Propositions can contain integers as well as floats. If an integer is compared to a floating point number, automatic conversion takes place, making the integer into a float.
