\paragraph{Floats}

Numbers with a decimal point, or \emph{floating point numbers}, generally work as you would expect from mathematics. Notably, dividing floats like \texttt{1.0\,/\,2.0} gives \texttt{0.5}.

\paragraph{Automatic conversion}

In expressions you may encounter combinations of floats and integers. This requires a decision as to what rule to apply. Often, this is solved by doing automatic conversion: if only a single float is involved, the other number (an integer) will be converted to a float, too. The result of the operation will then also be a float.

Note that precedence rules still define what happens first: the expression \texttt{3\,/\,2\,+\,0.25} evaluates to \texttt{1.25}. First, the subexpression \texttt{3\,/\,2} is evaluated, giving the integer \texttt{1}. Then, \texttt{1} and \texttt{0.25} are to be added. Only then does the automatic conversion kick in, making the result a float: \texttt{1.25}.
