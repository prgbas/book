\paragraph{Parameters}

Functions often have \emph{parameters}. When calling the function, we supply \emph{concrete} values for these parameters, which is called \emph{parameter passing}. From the function's perspective, these values are assigned names the are specified in the \emph{parameter list} of the function definition.

\begin{verbatim}
<type> <name of function>(<parameter list>):
    <function body>
\end{verbatim}

Now consider these two function definitions. Both have two parameters named in the parameter list.

\begin{figure}[h]
\begin{subfigure}[b]{.5\linewidth}
\begin{verbatim}
void date_1(day, month):
    print(day)
    print(month)
\end{verbatim}
\end{subfigure}
\begin{subfigure}[b]{.5\linewidth}
\begin{verbatim}
void date_2(month, day):
    print(day)
    print(month)
\end{verbatim}
\end{subfigure}
\end{figure}

In the left definition we specify the parameters \texttt{day} and \texttt{month}. In the right definition, we specify \texttt{month} and \texttt{day}, so in the reverse order. This \emph{order} has an effect on what names are given to the concrete parameters that are passed when calling the function. Let's call the functions:

\begin{figure}[h]
\begin{subfigure}[b]{.5\linewidth}
\begin{verbatim}
date_1(21, 6)
\end{verbatim}
\end{subfigure}
\begin{subfigure}[b]{.5\linewidth}
\begin{verbatim}
date_2(6, 21)
\end{verbatim}
\end{subfigure}
\end{figure}

The output of the functions would then be:

\begin{figure}[h]
\begin{subfigure}[b]{.5\linewidth}
\begin{verbatim}
21
6
\end{verbatim}
\end{subfigure}
\begin{subfigure}[b]{.5\linewidth}
\begin{verbatim}
21
6
\end{verbatim}
\end{subfigure}
\end{figure}

\paragraph{Tracing}

Keeping track of all values when passing parameters can easily become very tedious, which is why we often need to trace them explicitly. Below, like before, we put a line next to the one function that is defined. We also mark the starting line with a little arrow. 

\includegraphics[width=.8\textwidth]{2-trace-params.jpeg}

On that starting line, the function \texttt{ash} is called. To the right of the definition of that function, we draw a line, and copy the function definition, while substituting the concrete values from the function call.

Combining the original function definition and the substituted version, we can infer that in the function \texttt{y = 12}, \texttt{x = 11} and \texttt{z = 10}. We use this information to substitute those values in the right places of the \texttt{print} line.

Then, only a small calculation is left, of which the result will be printed. When we evaluate the expression using the basic rules, we get the number that will be printed.
