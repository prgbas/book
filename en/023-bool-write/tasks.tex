In the previous sections, you have \emph{evaluated} expressions, calculating the outcome by applying rules. Here, we ask you to write expressions yourself. This is a more creative endeavour, and it may be hard to find the ``right idea''. In that case, have a look at the previous sections for inspiration.

When writing such an expression, you can check it by applying the rules again. Write down your expression and try to evaluate it. You can even come up with a couple of concrete ``test cases'', filling in the number \texttt{n} to see if the result is what you expect.

\begin{exercise}
    \begin{longtasks}[resume=true](1)
        \task
        Write an expression that tests if a number \texttt{n} is smaller than 5.
        \task
        Write an expression that tests if a number \texttt{n} is between 5 and 10 (both 5 and 10 included!).
        \task
        Write an expression that tests if a number \texttt{n} is divisible by 3.
        \task
        Write an expression that tests if a number \texttt{n} is \emph{even}.
        \task
        Write an expression that tests if a number \texttt{n} is \emph{odd}.
        \task
        For which values does the expression \texttt{!(n > 1 \&\& !(n > 3)) \&\& true} yield \texttt{true}?
        \task
        The expression from the previous question may be written in more simply. Do this, and check if your simplified version is indeed equivalent to the original by checking if both yield the same value when substituting different values for \texttt{n}.
    \end{longtasks}
\end{exercise}

\begin{solution}
    No solutions are provided for section 2.3.
\end{solution}

% \begin{solution}
%     \begin{mltasks}(2)
%         \task \prglisting{071-sequences/01s.code}
%         \task \prglisting{071-sequences/02s.code}
%         \task \prglisting{071-sequences/03s.code}
%         \task \prglisting{071-sequences/04s.code}
%         \task \prglisting{071-sequences/05s.code}
%         \task \prglisting{071-sequences/06s.code}
%         \task \prglisting{071-sequences/07s.code}
%     \end{mltasks}
% \end{solution}


% \begin{solution}
% \texttt{n} < 5
% \end{solution}

% \begin{solution}
% 5 <= \texttt{n} <= 10
% \end{solution}

% \begin{solution}
% \texttt{n} % 3 == 0
% \end{solution}

% \begin{solution}
% \texttt{n} % 2 == 0
% \end{solution}

% \begin{solution}
% \texttt{n} % 2 != 0
% \end{solution}

% \begin{solution}
% n <= 1 \& n > 3
% \end{solution}

% \begin{solution}
% !(n >= 1 \&\& n < 3)
% \end{solution}
