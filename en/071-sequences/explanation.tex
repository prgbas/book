\paragraph{Generating sequences}

You can use a loop to print number sequences. This is a \texttt{for}-loop that prints the numbers from 0 up to and including 5:

% \vspace{-\abovedisplayskip}
\begin{minipage}[t]{0.5\textwidth}
\begin{nllisting}
for(i = 0; i < 6; i++)
    print(i)
\end{nllisting}
\end{minipage}%
\begin{minipage}[t]{0.5\textwidth}
\begin{bklisting}
0
1
2
3
4
5
\end{bklisting}
\end{minipage}

To print other sequences we need to find out the rhythm of the sequence. Quite often, you can base your loop on the example above. Let's say we would like to print six numbers from the sequence $0, 4, 8, 12, ...$, we might notice that each of those numbers is exactly four times the number from the earlier sequence. This means that we could adapt our code as follows:

\begin{minipage}[t]{0.5\textwidth}
\begin{nllisting}
for(i = 0; i < 10; i++)
    print(i * 4)
\end{nllisting}
\end{minipage}%
\begin{minipage}[t]{0.5\textwidth}
\begin{bklisting}
0
4
8
12
16
20
\end{bklisting}
\end{minipage}

If we would like to print the sequence $1, 2, 3, 4, ...$, starting with 1 instead of 0, we could use the following code. Compared to the first example, each number is \emph{one more}:

\begin{minipage}[t]{0.5\textwidth}
\begin{nllisting}
for(i = 0; i < 6; i++)
    print(i + 1)
\end{nllisting}
\end{minipage}
\begin{minipage}[t]{0.5\textwidth}
\begin{bklisting}
1
2
3
4
5
6
\end{bklisting}
\end{minipage}

\paragraph{Writing an algorithm}

To design an algorithm is to create a general procedure that can be easily changed to accommodate different situations. In this case, there are two things that can be changed. The \emph{rhythm} of the sequence is defined by the formula in the \texttt{print}-statement. The \emph{length} of the sequence can be controlled by changing the loop condition. By using  \texttt{i < 6} we have been printing sequences of six numbers, but you can change it as required.

\paragraph{Tracing}

To check the correctness of your programs, you can trace them after writing. Most likely, you will not even have to trace the full extent of the program. Instead, tracing a couple of steps is enough, as long as you can be sure that the rhythm of the generated sequence is as expected.


% Laten we eens kijken hoe we je met een \emph{loop} verschillende getallenreeksen kunt printen. Stel je wil de eerste \texttt{10} getallen van de volgende reeks printen:
% \begin{verbatim}
% 0 4 8 12 16 20 ...
% \end{verbatim}
% Aangezien we weten hoeveel getallen we willen printen, namelijk \texttt{10}, ligt een \texttt{for}-loop voor de hand als keuze (maar je zou natuurlijk ook een \texttt{while}-loop kunnen gebruiken). Laten we beginnen met het framework voor de \emph{loop}:
% \begin{verbatim}
% 1 for(i = 0; i < 10; i++)
% 2     ...
% \end{verbatim}
% We weten dat deze \emph{loop} 10 keer wordt herhaald, maar verder doet het nog niet zoveel. Op regel \texttt{2} weten we nog niet wat me moeten doen. Maar, we kunnen al wel een \emph{tijdelijke} trace\footnote{Door ruimtegebrek kunnen we maar vier van de tien iteraties traceren.} maken:
%
% \setlength{\tabcolsep}{5pt}
% \begin{tracelist-left}[l|cccccccccccccclc]
% regel & \texttt{1i} & \texttt{1c} & \texttt{2} & \texttt{1u} & \texttt{1c}
%                                   & \texttt{2} & \texttt{1u} & \texttt{1c}
%                                   & \texttt{2} & \texttt{1u} & \texttt{1c}
%                                   & \texttt{2} & \texttt{1u} & \texttt{1c} & ... \\ \hline
% var i & \fbox{\texttt{0}} & \texttt{0} & \texttt{0} & \fbox{\texttt{1}} & \texttt{1}
%                                   & \texttt{1} & \fbox{\texttt{2}} & \texttt{2}
%                                   & \texttt{2} & \fbox{\texttt{3}} & \texttt{3}
%                                   & \texttt{3} & \fbox{\texttt{4}} & \texttt{4} & ...
% \end{tracelist-left}
% \setlength{\tabcolsep}{6pt}
%
% Hoe verder? We weten dat we steeds een getal moeten printen. Dus met die kennis kunnen we de \emph{loop} vast wat verder invullen:
%
% \begin{verbatim}
% 1 for(i = 0; i < 10; i++)
% 2     getal = ???
% 3     print(getal)
% \end{verbatim}
%
% Maar, wat moeten we voor \texttt{getal} invullen? We maken eerst een \textbf{gewenste} trace met de getallen uit de reeks (\texttt{0 4 8 12 ...}):
%
% \setlength{\tabcolsep}{2.5pt}
% \begin{tracelist-left}[l|cccccccccccccccccclc]
% regel & \texttt{1i} & \texttt{1c} & \texttt{2} & \texttt{3} & \texttt{1u} & \texttt{1c}
%                                   & \texttt{2} & \texttt{3} & \texttt{1u} & \texttt{1c}
%                                   & \texttt{2} & \texttt{3} & \texttt{1u} & \texttt{1c}
%                                   & \texttt{2} & \texttt{3} & \texttt{1u} & \texttt{1c} & ... \\ \hline
% var i & \fbox{\texttt{0}} & \texttt{0} & \texttt{0} & \texttt{0} &  \fbox{\texttt{1}} & \texttt{1}
%                                   & \texttt{1} & \texttt{1} & \fbox{\texttt{2}} & \texttt{2}
%                                   & \texttt{2} & \texttt{2} & \fbox{\texttt{3}} & \texttt{3}
%                                   & \texttt{3} & \texttt{3} & \fbox{\texttt{4}} & \texttt{4} & ...\\
% var getal &  &  & \fbox{\texttt{0}} & \texttt{0} & \texttt{0} & \texttt{0}
%                 & \fbox{\texttt{4}} & \texttt{4} & \texttt{4} & \texttt{4}
%                 & \fbox{\texttt{8}} & \texttt{8} & \texttt{8} & \texttt{8}
%                 & \fbox{\texttt{12}} & \texttt{12} & \texttt{12} & \texttt{12} & ... \\
% print &  &  &  & \texttt{0} & &
%             & & \texttt{4} & &
%             & & \texttt{8} & &
%             & & \texttt{12} & & & ...
% \end{tracelist-left}
% \setlength{\tabcolsep}{6pt}
%
% Met deze gewenste trace kunnen we \textbf{terugredeneren} wat het bijbehorende algoritme zou moeten zijn: We hebben in deze trace ingevuld wat het algoritme zou moeten printen. Daaruit kunnen we vervolgens de waarde van \texttt{getal} afleiden. Nu is het makkelijk te zien dat de variabel \texttt{getal} steeds 4 keer zo groot moet zijn als de index \texttt{i}, oftewel \verb|getal = i * 4|. Als we dat in het algoritme invullen zijn we klaar:
%
% \begin{verbatim}
% 1 for(i = 0; i < 10; i++)
% 2     getal = i * 4
% 3     print(getal)
% \end{verbatim}
%
% Je kan het algoritme verifi"eren door zelf de trace na te lopen.
