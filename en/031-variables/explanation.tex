\paragraph{Assignment}

In the following code fragment, a \emph{value} is assigned to a \emph{variable}:

\begin{tracelist}
  het = 2.2
\end{tracelist}

Upon executing that line of code, a variable is created with the name \texttt{het}, and it is assigned the value \texttt{2.2}. This variable then becomes part of the \emph{final state} after executing the code fragment. However, a code fragment can also contain multiple assignments below each other. In the following example, we assign three variables, each with their own name:

\begin{tracelist}[l|ccc]
                     &         \texttt{ike} &      \texttt{dev} &         \texttt{wan} \\
 \texttt{ike = 3.14} & \fbox{\texttt{3.14}} &                   &                      \\
\texttt{dev = 3 / 4} &        \texttt{3.14} & \fbox{\texttt{0}} &                      \\
 \texttt{wan = 0.75} &        \texttt{3.14} &       \texttt{0} & \fbox{\texttt{0.75}}
\end{tracelist}

On the left we show the lines of code that are executed. On the right, we keep track of what happens when executing each line: we \emph{trace} the code fragment. As one variable is being assigned, we draw a box around the new value. On the lines below, that value is retained, which we show by copying the value down. By doing this for all lines, we can read the final state of all variables on the last line: \texttt{ike}, \texttt{dev} and \texttt{wan}, with their accompanying values.

\paragraph{Order}

It's possible to assign a value to a variable for a second time (or more often). The ``old'' value will be \emph{overwritten}. This makes \emph{order} of the program important: we always process the lines from top to bottom. Take a look at the following example.

\begin{tracelist}[l|c]
          & wei               \\
  wei = 1 & \cancel{\fbox{1}} \\
  wei = 4 & \fbox{4}
\end{tracelist}

The variable \texttt{wei} is assigned a value two times, as is shown by the two boxes that are drawn around the values. But the final state only consists of a single variable named \texttt{wei}, with value \texttt{4}. The value \texttt{1} that was assigned earlier has disappeared when it was overwritten.

\paragraph{Variables in expressions}

Now that we have variables, we can also \emph{use} them in calculations, referring to them by their name.

\begin{tracelist}[l|cc]
                & hat      & say      \\
  hat = 1       & \fbox{1} &          \\
  say = hat + 4 & 1        & \fbox{5}
\end{tracelist}

The state after executing the final line of code consists of two variables: \texttt{hat = 1} and \texttt{say = 5}.
